%% Generated by Sphinx.
\def\sphinxdocclass{report}
\documentclass[letterpaper,10pt,english]{sphinxmanual}
\ifdefined\pdfpxdimen
   \let\sphinxpxdimen\pdfpxdimen\else\newdimen\sphinxpxdimen
\fi \sphinxpxdimen=.75bp\relax

\PassOptionsToPackage{warn}{textcomp}
\usepackage[utf8]{inputenc}
\ifdefined\DeclareUnicodeCharacter
% support both utf8 and utf8x syntaxes
  \ifdefined\DeclareUnicodeCharacterAsOptional
    \def\sphinxDUC#1{\DeclareUnicodeCharacter{"#1}}
  \else
    \let\sphinxDUC\DeclareUnicodeCharacter
  \fi
  \sphinxDUC{00A0}{\nobreakspace}
  \sphinxDUC{2500}{\sphinxunichar{2500}}
  \sphinxDUC{2502}{\sphinxunichar{2502}}
  \sphinxDUC{2514}{\sphinxunichar{2514}}
  \sphinxDUC{251C}{\sphinxunichar{251C}}
  \sphinxDUC{2572}{\textbackslash}
\fi
\usepackage{cmap}
\usepackage[T1]{fontenc}
\usepackage{amsmath,amssymb,amstext}
\usepackage{babel}



\usepackage{times}
\expandafter\ifx\csname T@LGR\endcsname\relax
\else
% LGR was declared as font encoding
  \substitutefont{LGR}{\rmdefault}{cmr}
  \substitutefont{LGR}{\sfdefault}{cmss}
  \substitutefont{LGR}{\ttdefault}{cmtt}
\fi
\expandafter\ifx\csname T@X2\endcsname\relax
  \expandafter\ifx\csname T@T2A\endcsname\relax
  \else
  % T2A was declared as font encoding
    \substitutefont{T2A}{\rmdefault}{cmr}
    \substitutefont{T2A}{\sfdefault}{cmss}
    \substitutefont{T2A}{\ttdefault}{cmtt}
  \fi
\else
% X2 was declared as font encoding
  \substitutefont{X2}{\rmdefault}{cmr}
  \substitutefont{X2}{\sfdefault}{cmss}
  \substitutefont{X2}{\ttdefault}{cmtt}
\fi


\usepackage[Bjarne]{fncychap}
\usepackage{sphinx}

\fvset{fontsize=\small}
\usepackage{geometry}

% Include hyperref last.
\usepackage{hyperref}
% Fix anchor placement for figures with captions.
\usepackage{hypcap}% it must be loaded after hyperref.
% Set up styles of URL: it should be placed after hyperref.
\urlstyle{same}

\usepackage{sphinxmessages}
\setcounter{tocdepth}{1}


% Jupyter Notebook code cell colors
\definecolor{nbsphinxin}{HTML}{307FC1}
\definecolor{nbsphinxout}{HTML}{BF5B3D}
\definecolor{nbsphinx-code-bg}{HTML}{F5F5F5}
\definecolor{nbsphinx-code-border}{HTML}{E0E0E0}
\definecolor{nbsphinx-stderr}{HTML}{FFDDDD}
% ANSI colors for output streams and traceback highlighting
\definecolor{ansi-black}{HTML}{3E424D}
\definecolor{ansi-black-intense}{HTML}{282C36}
\definecolor{ansi-red}{HTML}{E75C58}
\definecolor{ansi-red-intense}{HTML}{B22B31}
\definecolor{ansi-green}{HTML}{00A250}
\definecolor{ansi-green-intense}{HTML}{007427}
\definecolor{ansi-yellow}{HTML}{DDB62B}
\definecolor{ansi-yellow-intense}{HTML}{B27D12}
\definecolor{ansi-blue}{HTML}{208FFB}
\definecolor{ansi-blue-intense}{HTML}{0065CA}
\definecolor{ansi-magenta}{HTML}{D160C4}
\definecolor{ansi-magenta-intense}{HTML}{A03196}
\definecolor{ansi-cyan}{HTML}{60C6C8}
\definecolor{ansi-cyan-intense}{HTML}{258F8F}
\definecolor{ansi-white}{HTML}{C5C1B4}
\definecolor{ansi-white-intense}{HTML}{A1A6B2}
\definecolor{ansi-default-inverse-fg}{HTML}{FFFFFF}
\definecolor{ansi-default-inverse-bg}{HTML}{000000}

% Define an environment for non-plain-text code cell outputs (e.g. images)
\makeatletter
\newenvironment{nbsphinxfancyoutput}{%
    % Avoid fatal error with framed.sty if graphics too long to fit on one page
    \let\sphinxincludegraphics\nbsphinxincludegraphics
    \nbsphinx@image@maxheight\textheight
    \advance\nbsphinx@image@maxheight -2\fboxsep   % default \fboxsep 3pt
    \advance\nbsphinx@image@maxheight -2\fboxrule  % default \fboxrule 0.4pt
    \advance\nbsphinx@image@maxheight -\baselineskip
\def\nbsphinxfcolorbox{\spx@fcolorbox{nbsphinx-code-border}{white}}%
\def\FrameCommand{\nbsphinxfcolorbox\nbsphinxfancyaddprompt\@empty}%
\def\FirstFrameCommand{\nbsphinxfcolorbox\nbsphinxfancyaddprompt\sphinxVerbatim@Continues}%
\def\MidFrameCommand{\nbsphinxfcolorbox\sphinxVerbatim@Continued\sphinxVerbatim@Continues}%
\def\LastFrameCommand{\nbsphinxfcolorbox\sphinxVerbatim@Continued\@empty}%
\MakeFramed{\advance\hsize-\width\@totalleftmargin\z@\linewidth\hsize\@setminipage}%
}{\par\unskip\@minipagefalse\endMakeFramed}
\makeatother
\newbox\nbsphinxpromptbox
\def\nbsphinxfancyaddprompt{\ifvoid\nbsphinxpromptbox\else
    \kern\fboxrule\kern\fboxsep
    \copy\nbsphinxpromptbox
    \kern-\ht\nbsphinxpromptbox\kern-\dp\nbsphinxpromptbox
    \kern-\fboxsep\kern-\fboxrule\nointerlineskip
    \fi}
\newlength\nbsphinxcodecellspacing
\setlength{\nbsphinxcodecellspacing}{0pt}

% Define support macros for attaching opening and closing lines to notebooks
\newsavebox\nbsphinxbox
\makeatletter
\newcommand{\nbsphinxstartnotebook}[1]{%
    \par
    % measure needed space
    \setbox\nbsphinxbox\vtop{{#1\par}}
    % reserve some space at bottom of page, else start new page
    \needspace{\dimexpr2.5\baselineskip+\ht\nbsphinxbox+\dp\nbsphinxbox}
    % mimick vertical spacing from \section command
      \addpenalty\@secpenalty
      \@tempskipa 3.5ex \@plus 1ex \@minus .2ex\relax
      \addvspace\@tempskipa
      {\Large\@tempskipa\baselineskip
             \advance\@tempskipa-\prevdepth
             \advance\@tempskipa-\ht\nbsphinxbox
             \ifdim\@tempskipa>\z@
               \vskip \@tempskipa
             \fi}
    \unvbox\nbsphinxbox
    % if notebook starts with a \section, prevent it from adding extra space
    \@nobreaktrue\everypar{\@nobreakfalse\everypar{}}%
    % compensate the parskip which will get inserted by next paragraph
    \nobreak\vskip-\parskip
    % do not break here
    \nobreak
}% end of \nbsphinxstartnotebook

\newcommand{\nbsphinxstopnotebook}[1]{%
    \par
    % measure needed space
    \setbox\nbsphinxbox\vbox{{#1\par}}
    \nobreak % it updates page totals
    \dimen@\pagegoal
    \advance\dimen@-\pagetotal \advance\dimen@-\pagedepth
    \advance\dimen@-\ht\nbsphinxbox \advance\dimen@-\dp\nbsphinxbox
    \ifdim\dimen@<\z@
      % little space left
      \unvbox\nbsphinxbox
      \kern-.8\baselineskip
      \nobreak\vskip\z@\@plus1fil
      \penalty100
      \vskip\z@\@plus-1fil
      \kern.8\baselineskip
    \else
      \unvbox\nbsphinxbox
    \fi
}% end of \nbsphinxstopnotebook

% Ensure height of an included graphics fits in nbsphinxfancyoutput frame
\newdimen\nbsphinx@image@maxheight % set in nbsphinxfancyoutput environment
\newcommand*{\nbsphinxincludegraphics}[2][]{%
    \gdef\spx@includegraphics@options{#1}%
    \setbox\spx@image@box\hbox{\includegraphics[#1,draft]{#2}}%
    \in@false
    \ifdim \wd\spx@image@box>\linewidth
      \g@addto@macro\spx@includegraphics@options{,width=\linewidth}%
      \in@true
    \fi
    % no rotation, no need to worry about depth
    \ifdim \ht\spx@image@box>\nbsphinx@image@maxheight
      \g@addto@macro\spx@includegraphics@options{,height=\nbsphinx@image@maxheight}%
      \in@true
    \fi
    \ifin@
      \g@addto@macro\spx@includegraphics@options{,keepaspectratio}%
    \fi
    \setbox\spx@image@box\box\voidb@x % clear memory
    \expandafter\includegraphics\expandafter[\spx@includegraphics@options]{#2}%
}% end of "\MakeFrame"-safe variant of \sphinxincludegraphics
\makeatother



\title{XIGrM}
\date{Sep 03, 2019}
\release{}
\author{Zhiwei Shao, Ziqian Hua, Douglas Rennehan}
\newcommand{\sphinxlogo}{\vbox{}}
\renewcommand{\releasename}{}
\makeindex
\begin{document}

\pagestyle{empty}
\sphinxmaketitle
\pagestyle{plain}
\sphinxtableofcontents
\pagestyle{normal}
\phantomsection\label{\detokenize{index::doc}}



\chapter{Introduction}
\label{\detokenize{index:introduction}}
This is a package for systematically analysing the X-ray properties
of cosmological simulations based on \sphinxhref{http://pynbody.github.io/pynbody/}{pynbody}
and \sphinxhref{https://academic.oup.com/mnras/article/456/4/4266/2892203}{Liang et al., 2016}.


\chapter{Contents}
\label{\detokenize{index:contents}}

\section{Pyatomdb Module}
\label{\detokenize{prepare_pyatomdb:module-modules.prepare_pyatomdb}}\label{\detokenize{prepare_pyatomdb:pyatomdb-module}}\label{\detokenize{prepare_pyatomdb::doc}}\index{modules.prepare\_pyatomdb (module)@\spxentry{modules.prepare\_pyatomdb}\spxextra{module}}
Generate a series of cooling rates and spectra 
tabulated with different atomic numbers (axis 0) 
and temperatures (axis 1).

Currently used atomdb version: 3.0.9

The temporary version only provides atomic data
within {[}0.01, 100{]} keV. To check the details, 
see 2.0.2 release notes in 
\sphinxurl{http://www.atomdb.org/download.php}
\index{AG89\_abundances() (in module modules.prepare\_pyatomdb)@\spxentry{AG89\_abundances()}\spxextra{in module modules.prepare\_pyatomdb}}

\begin{fulllineitems}
\phantomsection\label{\detokenize{prepare_pyatomdb:modules.prepare_pyatomdb.AG89_abundances}}\pysiglinewithargsret{\sphinxcode{\sphinxupquote{modules.prepare\_pyatomdb.}}\sphinxbfcode{\sphinxupquote{AG89\_abundances}}}{\emph{atomic\_numbers}}{}
Get AG89 abundances of the given input atomic numbers.
Based on pyatomdb.atomdb.get\_abundance().

\end{fulllineitems}

\index{calculate\_continuum\_emission() (in module modules.prepare\_pyatomdb)@\spxentry{calculate\_continuum\_emission()}\spxextra{in module modules.prepare\_pyatomdb}}

\begin{fulllineitems}
\phantomsection\label{\detokenize{prepare_pyatomdb:modules.prepare_pyatomdb.calculate_continuum_emission}}\pysiglinewithargsret{\sphinxcode{\sphinxupquote{modules.prepare\_pyatomdb.}}\sphinxbfcode{\sphinxupquote{calculate\_continuum\_emission}}}{\emph{energy\_bins}, \emph{specific\_elements=array({[} 1}, \emph{2}, \emph{6}, \emph{7}, \emph{8}, \emph{10}, \emph{12}, \emph{14}, \emph{16}, \emph{20}, \emph{26{]})}, \emph{return\_spectra=False}}{}
Calculate continuum emissions and cooling rates 
for individual atoms in atomdb.
\begin{quote}\begin{description}
\item[{Parameters}] \leavevmode\begin{description}
\item[{\sphinxstylestrong{energy\_bins}}] \leavevmode
Energy\_bins to calculate cooling rates and 
generate spectra on, must be in keV in the 
range of {[}0.01, 100{]}.

\item[{\sphinxstylestrong{specfic\_elements}}] \leavevmode
Atomic numbers of elements to be individually 
listed in the result. All the other elements 
in atomdb will also be calculated but will be 
added together as the last element of the result.

\item[{\sphinxstylestrong{return\_spectra}}] \leavevmode
Whether to return generated spectra.

\end{description}

\item[{Returns}] \leavevmode\begin{description}
\item[{dict}] \leavevmode
A dictionary consists of Cooling rates (key: ‘CoolingRate’) 
and spectra (key: ‘Emissivity’) if chosen contributed 
by continuum for different elements at different temperatures.

\end{description}

\end{description}\end{quote}

\end{fulllineitems}

\index{calculate\_line\_emission() (in module modules.prepare\_pyatomdb)@\spxentry{calculate\_line\_emission()}\spxextra{in module modules.prepare\_pyatomdb}}

\begin{fulllineitems}
\phantomsection\label{\detokenize{prepare_pyatomdb:modules.prepare_pyatomdb.calculate_line_emission}}\pysiglinewithargsret{\sphinxcode{\sphinxupquote{modules.prepare\_pyatomdb.}}\sphinxbfcode{\sphinxupquote{calculate\_line\_emission}}}{\emph{energy\_bins}, \emph{specific\_elements=array({[} 1}, \emph{2}, \emph{6}, \emph{7}, \emph{8}, \emph{10}, \emph{12}, \emph{14}, \emph{16}, \emph{20}, \emph{26{]})}, \emph{return\_spectra=False}}{}
Calculate line emissions and cooling rates for 
individual atoms in atomdb.
\begin{quote}\begin{description}
\item[{Parameters}] \leavevmode\begin{description}
\item[{\sphinxstylestrong{energy\_bins}}] \leavevmode
Energy\_bins to calculate cooling rates and 
generate spectra on, must be in keV in the 
range of {[}0.01, 100{]}.

\item[{\sphinxstylestrong{specfic\_elements}}] \leavevmode
Atomic numbers of elements to be individually 
listed in the result. All the other elements 
in atomdb will also be calculated but will be 
added together as the last element of the result.

\item[{\sphinxstylestrong{return\_spectra}}] \leavevmode
Whether to return generated spectra.

\end{description}

\item[{Returns}] \leavevmode\begin{description}
\item[{dict}] \leavevmode
A dictionary consists of Cooling rates (key: ‘CoolingRate’) 
and spectra (key: ‘Emissivity’) if chosen contributed by emission 
lines for different elements at different temperatures.

\end{description}

\end{description}\end{quote}

\end{fulllineitems}

\index{elsymbs\_to\_z0s() (in module modules.prepare\_pyatomdb)@\spxentry{elsymbs\_to\_z0s()}\spxextra{in module modules.prepare\_pyatomdb}}

\begin{fulllineitems}
\phantomsection\label{\detokenize{prepare_pyatomdb:modules.prepare_pyatomdb.elsymbs_to_z0s}}\pysiglinewithargsret{\sphinxcode{\sphinxupquote{modules.prepare\_pyatomdb.}}\sphinxbfcode{\sphinxupquote{elsymbs\_to\_z0s}}}{\emph{elements}}{}
Convert element symbols to atomic numbers.
Based on pyatomdb.atomic.elsymb\_to\_z0().

\end{fulllineitems}

\index{get\_atomic\_masses() (in module modules.prepare\_pyatomdb)@\spxentry{get\_atomic\_masses()}\spxextra{in module modules.prepare\_pyatomdb}}

\begin{fulllineitems}
\phantomsection\label{\detokenize{prepare_pyatomdb:modules.prepare_pyatomdb.get_atomic_masses}}\pysiglinewithargsret{\sphinxcode{\sphinxupquote{modules.prepare\_pyatomdb.}}\sphinxbfcode{\sphinxupquote{get\_atomic\_masses}}}{\emph{atomic\_numbers}}{}
Get atomic masses of the input atomic numbers.
Based on pyatomdb.atomic.Z\_to\_mass().

\end{fulllineitems}

\index{get\_index() (in module modules.prepare\_pyatomdb)@\spxentry{get\_index()}\spxextra{in module modules.prepare\_pyatomdb}}

\begin{fulllineitems}
\phantomsection\label{\detokenize{prepare_pyatomdb:modules.prepare_pyatomdb.get_index}}\pysiglinewithargsret{\sphinxcode{\sphinxupquote{modules.prepare\_pyatomdb.}}\sphinxbfcode{\sphinxupquote{get\_index}}}{\emph{te}, \emph{teunits='K'}, \emph{logscale=False}}{}
Finds indexes in the calculated table with kT closest ro desired kT.
\begin{quote}\begin{description}
\item[{Parameters}] \leavevmode\begin{description}
\item[{\sphinxstylestrong{te}}] \leavevmode{[}numpy.ndarray{]}
Temperatures in keV or K

\item[{\sphinxstylestrong{teunits}}] \leavevmode{[}\{‘keV’ , ‘K’\}{]}
Units of te (kev or K, default keV)

\item[{\sphinxstylestrong{logscale}}] \leavevmode{[}bool{]}
Search on a log scale for nearest temperature if set.

\end{description}

\item[{Returns}] \leavevmode\begin{description}
\item[{numpy.adarray}] \leavevmode
Indexes in the Temperature list.

\end{description}

\end{description}\end{quote}

\end{fulllineitems}

\index{load\_emissivity\_file() (in module modules.prepare\_pyatomdb)@\spxentry{load\_emissivity\_file()}\spxextra{in module modules.prepare\_pyatomdb}}

\begin{fulllineitems}
\phantomsection\label{\detokenize{prepare_pyatomdb:modules.prepare_pyatomdb.load_emissivity_file}}\pysiglinewithargsret{\sphinxcode{\sphinxupquote{modules.prepare\_pyatomdb.}}\sphinxbfcode{\sphinxupquote{load\_emissivity\_file}}}{\emph{filename, specific\_elements=None, energy\_band={[}0.5, 2.0{]}, n\_bins=1000}}{}
Load the emissivity file calculated based on pyatomdb.
If filename can’t be loaded, then will calculate and save
the emissivity information based on supplied specific\_elements.
\begin{quote}\begin{description}
\item[{Parameters}] \leavevmode\begin{description}
\item[{\sphinxstylestrong{filename}}] \leavevmode{[}str{]}
File name of the emissivity file to load.

\item[{\sphinxstylestrong{specific\_elements}}] \leavevmode
List of element symbols to include in calculation. 
If set to None, will automatically calculate all 
elements included in pyatomdb.

\item[{\sphinxstylestrong{energy\_band}}] \leavevmode
{[}min, max{]} energy range in keV to calculate emissivity within.

\item[{\sphinxstylestrong{n\_bins}}] \leavevmode{[}int{]}
Number of bins when calculating emissivity.

\end{description}

\end{description}\end{quote}

\end{fulllineitems}



\section{Cosmology Module}
\label{\detokenize{cosmology:module-modules.cosmology}}\label{\detokenize{cosmology:cosmology-module}}\label{\detokenize{cosmology::doc}}\index{modules.cosmology (module)@\spxentry{modules.cosmology}\spxextra{module}}
Calculate cosmological parameters.
\index{Delta\_vir() (in module modules.cosmology)@\spxentry{Delta\_vir()}\spxextra{in module modules.cosmology}}

\begin{fulllineitems}
\phantomsection\label{\detokenize{cosmology:modules.cosmology.Delta_vir}}\pysiglinewithargsret{\sphinxcode{\sphinxupquote{modules.cosmology.}}\sphinxbfcode{\sphinxupquote{Delta\_vir}}}{\emph{sim}}{}
Calculate the virial overdensity factor according
to eq. 3 in Liang et al. (2016) and hereafter.
\begin{quote}\begin{description}
\item[{Parameters}] \leavevmode\begin{description}
\item[{\sphinxstylestrong{sim}}] \leavevmode{[}pynbody.snapshot.SimSnap{]}
\end{description}

\end{description}\end{quote}

\end{fulllineitems}

\index{Ez() (in module modules.cosmology)@\spxentry{Ez()}\spxextra{in module modules.cosmology}}

\begin{fulllineitems}
\phantomsection\label{\detokenize{cosmology:modules.cosmology.Ez}}\pysiglinewithargsret{\sphinxcode{\sphinxupquote{modules.cosmology.}}\sphinxbfcode{\sphinxupquote{Ez}}}{\emph{sim}}{}
Calculate E(z)=H(z)/H\_0.
\begin{quote}\begin{description}
\item[{Parameters}] \leavevmode\begin{description}
\item[{\sphinxstylestrong{sim}}] \leavevmode{[}pynbody.snapshot.SimSnap{]}
\end{description}

\end{description}\end{quote}

\end{fulllineitems}



\section{Gas Properties Module}
\label{\detokenize{gas_properties:module-modules.gas_properties}}\label{\detokenize{gas_properties:gas-properties-module}}\label{\detokenize{gas_properties::doc}}\index{modules.gas\_properties (module)@\spxentry{modules.gas\_properties}\spxextra{module}}
Tools to generate basic information of gas particles
required by following analysing.
\index{abundance\_to\_solar() (in module modules.gas\_properties)@\spxentry{abundance\_to\_solar()}\spxextra{in module modules.gas\_properties}}

\begin{fulllineitems}
\phantomsection\label{\detokenize{gas_properties:modules.gas_properties.abundance_to_solar}}\pysiglinewithargsret{\sphinxcode{\sphinxupquote{modules.gas\_properties.}}\sphinxbfcode{\sphinxupquote{abundance\_to\_solar}}}{\emph{mass\_fraction, elements={[}'H', 'He', 'C', 'N', 'O', 'Ne', 'Mg', 'Si', 'S', 'Ca', 'Fe'{]}}}{}
Convert elements mass fraction to abundance 
relative to AG89 which is accepted by pyatomdb.
(AG89: Anders, E. and Grevesse, N. 1989, 
Geochimica et Cosmochimica Acta, 53, 197)
\begin{quote}\begin{description}
\item[{Parameters}] \leavevmode\begin{description}
\item[{\sphinxstylestrong{mass\_fraction}}] \leavevmode
Mass fractions of different elements. In the 
shape of (n\_particles, n\_elements). The order 
of element list must be sorted as atomic number 
from small to large. Hydrogen must be included.

\item[{\sphinxstylestrong{elements}}] \leavevmode
List of elements symbols included.

\end{description}

\item[{Returns}] \leavevmode\begin{description}
\item[{numpy.ndarrays}] \leavevmode
Abundance with respect to AG89 results. In the 
shape of (n\_particles, n\_elements).

\end{description}

\end{description}\end{quote}

\end{fulllineitems}

\index{calcu\_luminosity() (in module modules.gas\_properties)@\spxentry{calcu\_luminosity()}\spxextra{in module modules.gas\_properties}}

\begin{fulllineitems}
\phantomsection\label{\detokenize{gas_properties:modules.gas_properties.calcu_luminosity}}\pysiglinewithargsret{\sphinxcode{\sphinxupquote{modules.gas\_properties.}}\sphinxbfcode{\sphinxupquote{calcu\_luminosity}}}{\emph{gas, filename, mode='total', elements={[}'H', 'He', 'C', 'N', 'O', 'Ne', 'Mg', 'Si', 'S', 'Ca', 'Fe'{]}, band={[}0.5, 2.0{]}, bins=1000}}{}
Calculate X-ray luminosity of gas particles.
\begin{quote}\begin{description}
\item[{Parameters}] \leavevmode\begin{description}
\item[{\sphinxstylestrong{gas}}] \leavevmode{[}pynbody.snapshot.SimSnap{]}
SubSnap of gas particles to be calculated.

\item[{\sphinxstylestrong{filename}}] \leavevmode{[}emissivity file{]}
\item[{\sphinxstylestrong{mode}}] \leavevmode{[}str{]}
If set to ‘total’, both continuum and line emission 
will be taken into account. If set to ‘cont’, only 
continuum emission will be considered.

\item[{\sphinxstylestrong{elements, band, bins}}] \leavevmode
Required by prepare\_pyatomdb.load\_emissivity\_file(). 
See load\_emissivity\_file() docmentation for details.

\end{description}

\item[{Returns}] \leavevmode\begin{description}
\item[{list}] \leavevmode
List of luminosities.

\end{description}

\end{description}\end{quote}

\end{fulllineitems}

\index{nh() (in module modules.gas\_properties)@\spxentry{nh()}\spxextra{in module modules.gas\_properties}}

\begin{fulllineitems}
\phantomsection\label{\detokenize{gas_properties:modules.gas_properties.nh}}\pysiglinewithargsret{\sphinxcode{\sphinxupquote{modules.gas\_properties.}}\sphinxbfcode{\sphinxupquote{nh}}}{\emph{sim}}{}
Calculating hydrogen number density from density.

\end{fulllineitems}

\index{temp() (in module modules.gas\_properties)@\spxentry{temp()}\spxextra{in module modules.gas\_properties}}

\begin{fulllineitems}
\phantomsection\label{\detokenize{gas_properties:modules.gas_properties.temp}}\pysiglinewithargsret{\sphinxcode{\sphinxupquote{modules.gas\_properties.}}\sphinxbfcode{\sphinxupquote{temp}}}{\emph{sim}}{}
Convert internal energy to temperature, following 
the instructions from \sphinxhref{http://www.tapir.caltech.edu/~phopkins/Site/GIZMO\_files/gizmo\_documentation.html\#snaps}{GIZMO documentation}

\end{fulllineitems}



\section{X-ray Properties Module}
\label{\detokenize{X_properties:module-modules.X_properties}}\label{\detokenize{X_properties:x-ray-properties-module}}\label{\detokenize{X_properties::doc}}\index{modules.X\_properties (module)@\spxentry{modules.X\_properties}\spxextra{module}}
Tools for analysing halo X-ray properties using pynbody.
Assume any necessary quantity is already prepared as 
the derived arrays of the snapshot.
\index{cal\_tspec() (in module modules.X\_properties)@\spxentry{cal\_tspec()}\spxextra{in module modules.X\_properties}}

\begin{fulllineitems}
\phantomsection\label{\detokenize{X_properties:modules.X_properties.cal_tspec}}\pysiglinewithargsret{\sphinxcode{\sphinxupquote{modules.X\_properties.}}\sphinxbfcode{\sphinxupquote{cal\_tspec}}}{\emph{hdgas}, \emph{cal\_f}, \emph{datatype}}{}
Calculate the Tspec of hot diffuse gas particles.
\begin{quote}\begin{description}
\item[{Parameters}] \leavevmode\begin{description}
\item[{\sphinxstylestrong{hdgas}}] \leavevmode{[}pynbody.snapshot.SubSnap{]}
SubSnap of the hot diffuse gas.

\item[{\sphinxstylestrong{cal\_f}}] \leavevmode{[}str{]}
Calibration file.

\item[{\sphinxstylestrong{datatype}}] \leavevmode{[}str{]}
Simulation data type.

\end{description}

\end{description}\end{quote}

\end{fulllineitems}

\index{cal\_tweight() (in module modules.X\_properties)@\spxentry{cal\_tweight()}\spxextra{in module modules.X\_properties}}

\begin{fulllineitems}
\phantomsection\label{\detokenize{X_properties:modules.X_properties.cal_tweight}}\pysiglinewithargsret{\sphinxcode{\sphinxupquote{modules.X\_properties.}}\sphinxbfcode{\sphinxupquote{cal\_tweight}}}{\emph{halo}, \emph{weight\_type='Lx'}}{}
Calculate luminosity weighted or mass weighted temperatures.
\begin{quote}\begin{description}
\item[{Parameters}] \leavevmode\begin{description}
\item[{\sphinxstylestrong{halo}}] \leavevmode{[}pynbody.snapshot.SubSnap{]}
SubSnap of the halo.

\item[{\sphinxstylestrong{weight\_type}}] \leavevmode{[}str{]}
Type of weight to take. Related to the available properties
of the gas. Now available: luminousity weighted (starts 
with ‘l’) and mass weighted (starts with ‘m’)

\end{description}

\end{description}\end{quote}

\end{fulllineitems}



\section{Calculating Radii Module}
\label{\detokenize{calculate_R:module-modules.calculate_R}}\label{\detokenize{calculate_R:calculating-radii-module}}\label{\detokenize{calculate_R::doc}}\index{modules.calculate\_R (module)@\spxentry{modules.calculate\_R}\spxextra{module}}
This module contains codes for caluclating radii R200, 
R500, etc (and corresponding mass) of halos.

Based on Douglas Rennehan’s code on HaloAnalysis.
\index{get\_radius() (in module modules.calculate\_R)@\spxentry{get\_radius()}\spxextra{in module modules.calculate\_R}}

\begin{fulllineitems}
\phantomsection\label{\detokenize{calculate_R:modules.calculate_R.get_radius}}\pysiglinewithargsret{\sphinxcode{\sphinxupquote{modules.calculate\_R.}}\sphinxbfcode{\sphinxupquote{get\_radius}}}{\emph{halo}, \emph{overdensities=array({[}{]}}, \emph{dtype=float64)}, \emph{rho\_crit=None}, \emph{precision=0.01}, \emph{rmax=None}, \emph{cen=array({[}{]}}, \emph{dtype=float64)}, \emph{prop=None}}{}
Calculate different radii of a given halo with a decreasing sphere method.
\begin{quote}\begin{description}
\item[{Parameters}] \leavevmode\begin{description}
\item[{\sphinxstylestrong{halo}}] \leavevmode
Halo to be calculated, SimSnap in pynbody. 
Paramaters need to be in physical\_units.

\item[{\sphinxstylestrong{overdensity}}] \leavevmode
Overdensity factor \$Delta\$s. Must be a 
list!

\item[{\sphinxstylestrong{rho\_crit}}] \leavevmode
Critical density of the universe at the 
redshift of current SimSnap. Must be in 
units of Msol/kpc**3.

\item[{\sphinxstylestrong{precision}}] \leavevmode
Precision for calculating radius.

\item[{\sphinxstylestrong{rmax}}] \leavevmode
The radius at which start using decreasing sphere 
method. If 0, then use 2 * distance of 
farthest particle. If set, must be in units of 
kpc or convertible to kpc via pynbody.

\item[{\sphinxstylestrong{cen}}] \leavevmode
center coordinates of calculated halo. If not provided, 
then will load from property dictionary or calculate 
via pynbody.analysis.halo.center().

\item[{\sphinxstylestrong{prop}}] \leavevmode
halo.properties dictionary. If set to 0 then will 
automatically load it. Only used for getting boxsize 
and redshift.

\end{description}

\item[{Returns}] \leavevmode\begin{description}
\item[{mass}] \leavevmode
Dict of masses of the halo within calculated radii.

\item[{radius}] \leavevmode
Dict of radii of the halo at given overdensities.

\end{description}

\end{description}\end{quote}

\end{fulllineitems}

\index{get\_radius\_bisection() (in module modules.calculate\_R)@\spxentry{get\_radius\_bisection()}\spxextra{in module modules.calculate\_R}}

\begin{fulllineitems}
\phantomsection\label{\detokenize{calculate_R:modules.calculate_R.get_radius_bisection}}\pysiglinewithargsret{\sphinxcode{\sphinxupquote{modules.calculate\_R.}}\sphinxbfcode{\sphinxupquote{get\_radius\_bisection}}}{\emph{halo}, \emph{overdensities=array({[}{]}}, \emph{dtype=float64)}, \emph{rho\_crit=0}, \emph{precision=0.01}, \emph{prop=0}}{}
Calculate different radii of a given halo with a 
bisection method. This is a prototype without much 
optimization. And you will need to modify the source 
code to make it compatible with the module.
\begin{quote}\begin{description}
\item[{Parameters}] \leavevmode\begin{description}
\item[{\sphinxstylestrong{halo}}] \leavevmode\begin{description}
\item[{Halo to be calculated, SimSnap in pynbody.}] \leavevmode
Paramaters need to be in physical\_units.

\end{description}

\item[{\sphinxstylestrong{overdensity}}] \leavevmode
Overdensity factor \$Delta\$s.

\item[{\sphinxstylestrong{rho\_crit}}] \leavevmode
Critical density of the universe at the
redshift of current SimSnap. Must be in
units of halo{[}‘mass’{]}.units / 
halo{[}‘pos’{]}.units**3.

\item[{\sphinxstylestrong{precision}}] \leavevmode
Precision within which radii is calculated.

\end{description}

\item[{Returns}] \leavevmode\begin{description}
\item[{mass}] \leavevmode
Dict of masses of the halo within calculated radii.

\item[{radius}] \leavevmode
Dict of radii of the halo at given overdensities.

\end{description}

\end{description}\end{quote}

\end{fulllineitems}



\section{Halo Analysis Module}
\label{\detokenize{halo_analysis:module-modules.halo_analysis}}\label{\detokenize{halo_analysis:halo-analysis-module}}\label{\detokenize{halo_analysis::doc}}\index{modules.halo\_analysis (module)@\spxentry{modules.halo\_analysis}\spxextra{module}}
Tools for analysing halo properties. 
Assume any necessary basic quantities is already 
prepared as the derived arrays of the snapshot.
\index{get\_union() (in module modules.halo\_analysis)@\spxentry{get\_union()}\spxextra{in module modules.halo\_analysis}}

\begin{fulllineitems}
\phantomsection\label{\detokenize{halo_analysis:modules.halo_analysis.get_union}}\pysiglinewithargsret{\sphinxcode{\sphinxupquote{modules.halo\_analysis.}}\sphinxbfcode{\sphinxupquote{get\_union}}}{\emph{catalogue}, \emph{list}}{}
Calculate the union of the particles listed in the list.
\begin{quote}\begin{description}
\item[{Parameters}] \leavevmode\begin{description}
\item[{\sphinxstylestrong{catalogue}}] \leavevmode{[}pynbody.halo.HaloCatalogue{]}
\item[{\sphinxstylestrong{list}}] \leavevmode
List of halos in the catalogue to get union.

\end{description}

\end{description}\end{quote}

\end{fulllineitems}

\index{halo\_props (class in modules.halo\_analysis)@\spxentry{halo\_props}\spxextra{class in modules.halo\_analysis}}

\begin{fulllineitems}
\phantomsection\label{\detokenize{halo_analysis:modules.halo_analysis.halo_props}}\pysiglinewithargsret{\sphinxbfcode{\sphinxupquote{class }}\sphinxcode{\sphinxupquote{modules.halo\_analysis.}}\sphinxbfcode{\sphinxupquote{halo\_props}}}{\emph{halocatalogue, datatype, field=\{'L': {[}'x', 'x\_cont', 'xb', 'xb\_cont'{]}, 'M': {[}'vir', '200', '500', '2500', 'star200', 'gas200', 'bar200', 'ism200', 'cold200', 'igrm200', 'star500', 'gas500', 'bar500', 'ism500', 'cold500', 'igrm500', 'total\_star', 'self\_star'{]}, 'R': {[}'vir', '200', '500', '2500'{]}, 'S': {[}'500', '2500'{]}, 'T': {[}'x', 'x\_cont', 'mass', 'spec', 'spec\_corr', 'x\_corr', 'x\_corr\_cont', 'mass\_corr'{]}\}, host\_id\_of\_top\_level=0}}{}
Bases: \sphinxcode{\sphinxupquote{object}}

Systematically analyse the halo X-ray properties based 
on other modules.
\begin{quote}\begin{description}
\item[{Attributes}] \leavevmode\begin{description}
\item[{\sphinxstylestrong{datatype}}] \leavevmode{[}str{]}
A copy of the input type of simulation data.

\item[{\sphinxstylestrong{catalogue\_original}}] \leavevmode{[}pynbody.halo.HaloCatalogue{]}
The input halo catalogue.

\item[{\sphinxstylestrong{length}}] \leavevmode{[}length of the input catalogue.{]}
\item[{\sphinxstylestrong{host\_id\_of\_top\_level}}] \leavevmode
How catalogue record “hostHalo” for those halos 
without a host halo. Default is 0.

\item[{\sphinxstylestrong{errorlist}}] \leavevmode{[}list{]}
Record when the host halo ID of a certain subhalo is not 
recorded in the catalogue (weird but will happen in
ahf sometimes).

\item[{\sphinxstylestrong{rho\_crit}}] \leavevmode
Critical density of the current snapshot in Msol kpc**-3.

\item[{\sphinxstylestrong{ovdens}}] \leavevmode
Virial overdensity factor \$Delta\$ of the current snapshot.

\item[{\sphinxstylestrong{dict}}] \leavevmode{[}astropy.table.Table{]}
A copy of the halo.properties dictionary but in a table form
to make future reference more convenient.

\item[{\sphinxstylestrong{haloid}}] \leavevmode
List of halo\_id given by property dictionary.

\item[{\sphinxstylestrong{IDlist}}] \leavevmode
Table of halo\_id and corresponding \#ID given in the property 
dictionary.

\item[{\sphinxstylestrong{hostid}}] \leavevmode
List of the halo\_id of the host halo of each halo (originally 
recorded in the property dictionary in the form of \#ID).

\item[{\sphinxstylestrong{new\_catalogue}}] \leavevmode{[}dict{]}
The new catalogue which includes all the subhalo particles 
in its host halo. The keys of the dictionary are the indexes of 
halos in \sphinxtitleref{catalogue\_original}.

\item[{\sphinxstylestrong{prop}}] \leavevmode
Table of quantities corresponding to input field.

\item[{\sphinxstylestrong{host\_list}}] \leavevmode
List of host halos.

\item[{\sphinxstylestrong{tophost}}] \leavevmode
halo\_ids of the top-level host halo for each halo.

\item[{\sphinxstylestrong{children}}] \leavevmode{[}list of sets{]}
Each set corresponds to the one-level down children of each halo.

\item[{\sphinxstylestrong{galaxy\_list}}] \leavevmode
List of all galaxies (as long as n\_star \textgreater{} 0).

\item[{\sphinxstylestrong{lumi\_galaxy\_list}}] \leavevmode
List of all luminous galaxies (self\_m\_star \textgreater{} galaxy\_low\_limit).

\item[{\sphinxstylestrong{galaxies}}] \leavevmode{[}list of sets{]}
Each set corresponds to the embeded galaxies of each halo. All 
the subhalos will not be considered and will have an empty set. 
And for host halos it will include all the galaxies within it, 
including the galaxies actually embedded in the subhalo (i.e., 
the children of subhalo).

\item[{\sphinxstylestrong{lumi\_galaxies}}] \leavevmode
Each set corresponds to the embeded luminous galaxies of each 
halo. Same as \sphinxtitleref{galaxies}, only care about host halo and include 
all the luminous galaxies within.

\item[{\sphinxstylestrong{n\_lgal}}] \leavevmode
Number of total luminous galaxies embedded in each halo. Again, 
only care about host halos and the galaxies within subhalos 
(i.e., subhalos themselves) will also be taken into account.

\item[{\sphinxstylestrong{group\_list}}] \leavevmode
halo\_id of the halo identified as group in the catalogue.

\end{description}

\end{description}\end{quote}
\index{calcu\_entropy() (modules.halo\_analysis.halo\_props method)@\spxentry{calcu\_entropy()}\spxextra{modules.halo\_analysis.halo\_props method}}

\begin{fulllineitems}
\phantomsection\label{\detokenize{halo_analysis:modules.halo_analysis.halo_props.calcu_entropy}}\pysiglinewithargsret{\sphinxbfcode{\sphinxupquote{calcu\_entropy}}}{\emph{self, cal\_file, halo\_id\_list=array({[}{]}, dtype=float64), calcu\_field={[}'500', '2500'{]}, thickness=SimArray(1, 'kpc')}}{}
Calculate all entropy within a thin spherical shell 
centered at halo.
\begin{quote}\begin{description}
\item[{Parameters}] \leavevmode\begin{description}
\item[{\sphinxstylestrong{cal\_file}}] \leavevmode
Calibration file used for calculating Tspec.

\item[{\sphinxstylestrong{halo\_id\_list}}] \leavevmode
List of halo\_ids to calculate entropies. 
If set to None, then will use self.group\_list.

\item[{\sphinxstylestrong{calcu\_field}}] \leavevmode
Radii of the thin shell to calculate entropies.

\item[{\sphinxstylestrong{thickness}}] \leavevmode
Thickness of the spherical shell.

\end{description}

\end{description}\end{quote}

\end{fulllineitems}

\index{calcu\_radii\_masses() (modules.halo\_analysis.halo\_props method)@\spxentry{calcu\_radii\_masses()}\spxextra{modules.halo\_analysis.halo\_props method}}

\begin{fulllineitems}
\phantomsection\label{\detokenize{halo_analysis:modules.halo_analysis.halo_props.calcu_radii_masses}}\pysiglinewithargsret{\sphinxbfcode{\sphinxupquote{calcu\_radii\_masses}}}{\emph{self}, \emph{halo\_id\_list=array({[}{]}}, \emph{dtype=float64)}, \emph{rdict=None}, \emph{precision=0.01}, \emph{rmax=None}}{}
Calculate radii (Rvir, R200, etc) and corresponding masses.
\begin{quote}\begin{description}
\item[{Parameters}] \leavevmode\begin{description}
\item[{\sphinxstylestrong{halo\_id\_list}}] \leavevmode
List of halo\_ids to calculate radii and masses. 
If set to None, then will use self.group\_list.

\item[{\sphinxstylestrong{rdict}}] \leavevmode
names and values for overdensity factors. Default is: 
\{‘vir’: self.ovdens, ‘200’: 200, ‘500’: 500, ‘2500’: 2500\}

\item[{\sphinxstylestrong{precision}}] \leavevmode
Precision for calculate radius. See get\_index() in 
calculate\_R.py documentation for detail.

\item[{\sphinxstylestrong{rmax}}] \leavevmode
Maximum value for the shrinking sphere method. See 
get\_index() in calculate\_R.py documentation for detail.

\end{description}

\end{description}\end{quote}

\end{fulllineitems}

\index{calcu\_specific\_masses() (modules.halo\_analysis.halo\_props method)@\spxentry{calcu\_specific\_masses()}\spxextra{modules.halo\_analysis.halo\_props method}}

\begin{fulllineitems}
\phantomsection\label{\detokenize{halo_analysis:modules.halo_analysis.halo_props.calcu_specific_masses}}\pysiglinewithargsret{\sphinxbfcode{\sphinxupquote{calcu\_specific\_masses}}}{\emph{self, halo\_id\_list=array({[}{]}, dtype=float64), calcu\_field={[}'200', '500'{]}}}{}
Calculate some specific masses, such as baryon, IGrM, etc.
\begin{quote}\begin{description}
\item[{Parameters}] \leavevmode\begin{description}
\item[{\sphinxstylestrong{halo\_id\_list}}] \leavevmode
List of halo\_ids to calculate masses. 
If set to None, then will use self.group\_list.

\item[{\sphinxstylestrong{calcu\_field}}] \leavevmode
Radii to calculate specific masses within.

\end{description}

\end{description}\end{quote}

\end{fulllineitems}

\index{calcu\_temp\_lumi() (modules.halo\_analysis.halo\_props method)@\spxentry{calcu\_temp\_lumi()}\spxextra{modules.halo\_analysis.halo\_props method}}

\begin{fulllineitems}
\phantomsection\label{\detokenize{halo_analysis:modules.halo_analysis.halo_props.calcu_temp_lumi}}\pysiglinewithargsret{\sphinxbfcode{\sphinxupquote{calcu\_temp\_lumi}}}{\emph{self}, \emph{cal\_file}, \emph{halo\_id\_list=array({[}{]}}, \emph{dtype=float64)}, \emph{core\_corr\_factor=0.15}, \emph{calcu\_field='500'}}{}
Calculate all the temperatures and luminosities listed in
temp\_field and luminosity\_field.
\begin{quote}\begin{description}
\item[{Parameters}] \leavevmode\begin{description}
\item[{\sphinxstylestrong{cal\_file}}] \leavevmode
Calibration file used for calculating Tspec.

\item[{\sphinxstylestrong{halo\_id\_list}}] \leavevmode
List of halo\_ids to calculate temperatures 
and luminosities. If set to None, then will use 
self.group\_list.

\item[{\sphinxstylestrong{core\_corr\_factor}}] \leavevmode
Inner radius for calculating core-corrected 
temperatures. Gas particles within 
(core\_corr\_factor*R, R) will be used for calculation.

\item[{\sphinxstylestrong{calcu\_field}}] \leavevmode
Radius to calculate temperatures and luminosities 
within. Must be in radius\_field. Default: R\_500.

\end{description}

\end{description}\end{quote}

\end{fulllineitems}

\index{calcu\_tspec() (modules.halo\_analysis.halo\_props method)@\spxentry{calcu\_tspec()}\spxextra{modules.halo\_analysis.halo\_props method}}

\begin{fulllineitems}
\phantomsection\label{\detokenize{halo_analysis:modules.halo_analysis.halo_props.calcu_tspec}}\pysiglinewithargsret{\sphinxbfcode{\sphinxupquote{calcu\_tspec}}}{\emph{self}, \emph{cal\_file}, \emph{halo\_id\_list=array({[}{]}}, \emph{dtype=float64)}, \emph{core\_corr\_factor=0.15}, \emph{calcu\_field='500'}}{}
Calculate spectroscopic temperatures based on Douglas’s 
pytspec module.
\begin{quote}\begin{description}
\item[{Parameters}] \leavevmode\begin{description}
\item[{\sphinxstylestrong{cal\_file}}] \leavevmode
Calibration file used for calculating Tspec.

\item[{\sphinxstylestrong{halo\_id\_list}}] \leavevmode
List of halo\_ids to calculate temperatures and 
luminosities. If set to None, then will use self.group\_list.

\item[{\sphinxstylestrong{core\_corr\_factor}}] \leavevmode
Inner radius for calculating core-corrected temperatures. 
Gas particles within (core\_corr\_factor*R, R) will be used 
for calculation.

\item[{\sphinxstylestrong{calcu\_field}}] \leavevmode
Radius to calculate temperatures and luminosities within. 
Must be in radius\_field. Default: R\_500.

\end{description}

\end{description}\end{quote}

\end{fulllineitems}

\index{calcu\_tx\_lx() (modules.halo\_analysis.halo\_props method)@\spxentry{calcu\_tx\_lx()}\spxextra{modules.halo\_analysis.halo\_props method}}

\begin{fulllineitems}
\phantomsection\label{\detokenize{halo_analysis:modules.halo_analysis.halo_props.calcu_tx_lx}}\pysiglinewithargsret{\sphinxbfcode{\sphinxupquote{calcu\_tx\_lx}}}{\emph{self}, \emph{halo\_id\_list=array({[}{]}}, \emph{dtype=float64)}, \emph{core\_corr\_factor=0.15}, \emph{calcu\_field='500'}}{}
Calculate X-ray luminosities and emission weighted 
temperatures listed in temp\_field and luminosity\_field.
\begin{quote}\begin{description}
\item[{Parameters}] \leavevmode\begin{description}
\item[{\sphinxstylestrong{halo\_id\_list}}] \leavevmode
List of halo\_ids to calculate temperatures on. 
If set to None, then will use self.group\_list.

\item[{\sphinxstylestrong{core\_corr\_factor}}] \leavevmode
Inner radius for calculating core-corrected 
temperatures. Gas particles within 
(core\_corr\_factor*R, R) will be used for calculation.

\item[{\sphinxstylestrong{calcu\_field}}] \leavevmode
Radius to calculate temperatures and luminosities 
within. Must be in radius\_field. Default: R\_500.

\end{description}

\end{description}\end{quote}

\end{fulllineitems}

\index{get\_center() (modules.halo\_analysis.halo\_props method)@\spxentry{get\_center()}\spxextra{modules.halo\_analysis.halo\_props method}}

\begin{fulllineitems}
\phantomsection\label{\detokenize{halo_analysis:modules.halo_analysis.halo_props.get_center}}\pysiglinewithargsret{\sphinxbfcode{\sphinxupquote{get\_center}}}{\emph{self}}{}
Calculate the center of the halos if an ahfcatalogue is 
provided, then will automatically load the results in ahf. 
Otherwise it will try to calculate the center coordinates 
via gravitional potential or center of mass.
\subsubsection*{Notes}

Due to a bug in pynbody, calculating center of mass will 
lead to an incorrect result for the halos crossing the 
periodical boundary of the simulation box. Make sure pynbody 
has fixed it before you use.

\end{fulllineitems}

\index{get\_children() (modules.halo\_analysis.halo\_props method)@\spxentry{get\_children()}\spxextra{modules.halo\_analysis.halo\_props method}}

\begin{fulllineitems}
\phantomsection\label{\detokenize{halo_analysis:modules.halo_analysis.halo_props.get_children}}\pysiglinewithargsret{\sphinxbfcode{\sphinxupquote{get\_children}}}{\emph{self}}{}
Generate list of children (subhalos) for each halo.
Subhalo itself can also have children. And this list 
will not contain “grandchildren” (i.e., the children 
of children).

\end{fulllineitems}

\index{get\_galaxy() (modules.halo\_analysis.halo\_props method)@\spxentry{get\_galaxy()}\spxextra{modules.halo\_analysis.halo\_props method}}

\begin{fulllineitems}
\phantomsection\label{\detokenize{halo_analysis:modules.halo_analysis.halo_props.get_galaxy}}\pysiglinewithargsret{\sphinxbfcode{\sphinxupquote{get\_galaxy}}}{\emph{self}, \emph{g\_low\_limit}}{}
Generate list of galaxies for each host halo. The subsubhalo 
will also be included in the hosthalo galaxy list. And it won’t 
generate list for the subhalos even if there are galaxies within.

\end{fulllineitems}

\index{get\_group\_list() (modules.halo\_analysis.halo\_props method)@\spxentry{get\_group\_list()}\spxextra{modules.halo\_analysis.halo\_props method}}

\begin{fulllineitems}
\phantomsection\label{\detokenize{halo_analysis:modules.halo_analysis.halo_props.get_group_list}}\pysiglinewithargsret{\sphinxbfcode{\sphinxupquote{get\_group\_list}}}{\emph{self}, \emph{N\_galaxy}}{}
halo\_id of the halo identified as group in the catalogue.
\begin{quote}\begin{description}
\item[{Parameters}] \leavevmode\begin{description}
\item[{\sphinxstylestrong{N\_galaxy}}] \leavevmode{[}int{]}
Number of luminous galaxies above which host halos 
are considered as groups.

\end{description}

\end{description}\end{quote}

\end{fulllineitems}

\index{get\_new\_catalogue() (modules.halo\_analysis.halo\_props method)@\spxentry{get\_new\_catalogue()}\spxextra{modules.halo\_analysis.halo\_props method}}

\begin{fulllineitems}
\phantomsection\label{\detokenize{halo_analysis:modules.halo_analysis.halo_props.get_new_catalogue}}\pysiglinewithargsret{\sphinxbfcode{\sphinxupquote{get\_new\_catalogue}}}{\emph{self}}{}
Generate a new catalogue based on catalogue\_original, 
the new catalogue will include all the subhalo particles 
in its host halo.

\end{fulllineitems}

\index{init\_relationship() (modules.halo\_analysis.halo\_props method)@\spxentry{init\_relationship()}\spxextra{modules.halo\_analysis.halo\_props method}}

\begin{fulllineitems}
\phantomsection\label{\detokenize{halo_analysis:modules.halo_analysis.halo_props.init_relationship}}\pysiglinewithargsret{\sphinxbfcode{\sphinxupquote{init\_relationship}}}{\emph{self}, \emph{galaxy\_low\_limit}, \emph{N\_galaxy=3}}{}
Get basic information regarding groups, hosts, children, etc.

\end{fulllineitems}

\index{savedata() (modules.halo\_analysis.halo\_props method)@\spxentry{savedata()}\spxextra{modules.halo\_analysis.halo\_props method}}

\begin{fulllineitems}
\phantomsection\label{\detokenize{halo_analysis:modules.halo_analysis.halo_props.savedata}}\pysiglinewithargsret{\sphinxbfcode{\sphinxupquote{savedata}}}{\emph{self, filename, field=\{'R': {[}'vir', '200', '500', '2500'{]}, 'M': {[}'vir', '200', '500', '2500', 'star200', 'gas200', 'bar200', 'ism200', 'cold200', 'igrm200', 'star500', 'gas500', 'bar500', 'ism500', 'cold500', 'igrm500', 'total\_star', 'self\_star'{]}, 'T': {[}'x', 'x\_cont', 'mass', 'spec', 'spec\_corr', 'x\_corr', 'x\_corr\_cont', 'mass\_corr'{]}, 'S': {[}'500', '2500'{]}, 'L': {[}'x', 'x\_cont', 'xb', 'xb\_cont'{]}\}, halo\_id\_list=array({[}{]}, dtype=float64), units=\{'T': 'keV', 'L': 'erg s**-1', 'R': 'kpc', 'M': 'Msol', 'S': 'keV cm**2'\}}}{}
Save the data in hdf5 format. Will save halo\_id\_list 
(key: ‘halo\_id’) and the quantities listed in field.
\begin{quote}\begin{description}
\item[{Parameters}] \leavevmode\begin{description}
\item[{\sphinxstylestrong{filename}}] \leavevmode
Filename of the hdf5 file.

\item[{\sphinxstylestrong{field}}] \leavevmode
Type of information to save.

\item[{\sphinxstylestrong{halo\_id\_list}}] \leavevmode
List of halo\_ids to save.If set to None, 
then will use self.group\_list.

\item[{\sphinxstylestrong{units}}] \leavevmode
Convert the data into specified inits and save.

\end{description}

\end{description}\end{quote}

\end{fulllineitems}


\end{fulllineitems}



\section{How to Use}
\label{\detokenize{Description:how-to-use}}\label{\detokenize{Description::doc}}

\subsection{Data Structure}
\label{\detokenize{Description:data-structure}}
Most useful halo information is stored in the \sphinxtitleref{prop}
attribute of the \sphinxtitleref{class halo\_props}. And to modify what
to be calculated, one will need to provide a new field
dictionary similarly organized as the default field
dictionary. Note that quantities related to temperature
and luminosity are mostly hard coded, so you may need to
modify the source code to meet your own requirements.

Generally there are four types of properties stored in
the \sphinxtitleref{prop} attribute: R (radius), M (mass), T (temperature)
and S (entropy). And the following list explains the physical
meaning of the terms in the default field dictionary.

R:


\begin{savenotes}\sphinxattablestart
\centering
\begin{tabulary}{\linewidth}[t]{|T|T|}
\hline
\sphinxstyletheadfamily 
Key name
&\sphinxstyletheadfamily 
Description
\\
\hline
vir
&
Virial radius \(R_{vir}\)
\\
\hline
200
&
\(R_{200}\)
\\
\hline
500
&
\(R_{500}\)
\\
\hline
2500
&
\(R_{2500}\)
\\
\hline
\end{tabulary}
\par
\sphinxattableend\end{savenotes}

M (\(X = 200, 500\)):


\begin{savenotes}\sphinxattablestart
\centering
\begin{tabulary}{\linewidth}[t]{|T|T|}
\hline
\sphinxstyletheadfamily 
Key name
&\sphinxstyletheadfamily 
Description
\\
\hline
vir
&
Mass enclosed within \(R_{vir}\)
\\
\hline
200
&
Mass enclosed within \(R_{200}\)
\\
\hline
500
&
Mass enclosed within \(R_{500}\)
\\
\hline
2500
&
Mass enclosed within \(R_{2500}\)
\\
\hline
starX
&
Stellar mass enclosed within \(R_{X}\)
\\
\hline
gasX
&
Gas mass enclosed within \(R_{X}\)
\\
\hline
barX
&
Baryon mass enclosed within \(R_{X}\)
\\
\hline
ismX
&
ISM mass enclosed within \(R_{X}\)
\\
\hline
coldX
&
Cold gas mass enclosed within \(R_{X}\)
\\
\hline
igrmX
&
IGrM mass enclosed within \(R_{X}\)
\\
\hline
total\_star
&
Stellar mass within halo, including contribution from subhalo
\\
\hline
self\_star
&
Stellar mass within halo, excluding contribution from subhalo
\\
\hline
\end{tabulary}
\par
\sphinxattableend\end{savenotes}

Before introducing temperature and entropy, I will first talk about the
luminosities used during calculation:


\begin{savenotes}\sphinxattablestart
\centering
\begin{tabulary}{\linewidth}[t]{|T|T|T|}
\hline
\sphinxstyletheadfamily 
Name
&\sphinxstyletheadfamily 
Key name
&\sphinxstyletheadfamily 
Description
\\
\hline
\(L_X\)
&
Lx
&
0.5-2.0keV X-ray luminosity, including both continuum and line emission
\\
\hline
\(L_{X, cont}\)
&
Lx\_cont
&
0.5-2.0keV X-ray luminosity, only including continuum emission
\\
\hline
\(L_{X, broad}\)
&
Lxb
&
0.1-2.4keV X-ray luminosity, including both continuum and line emission
\\
\hline
\(L_{X, broad, cont}\)
&
Lxb\_cont
&
0.1-2.4keV X-ray luminosity, only including continuum emission
\\
\hline
\end{tabulary}
\par
\sphinxattableend\end{savenotes}

T (only hot diffuse gas (IGrM) is taken into account when calculating temperature):


\begin{savenotes}\sphinxattablestart
\centering
\begin{tabulary}{\linewidth}[t]{|T|T|}
\hline
\sphinxstyletheadfamily 
Key name
&\sphinxstyletheadfamily 
Description
\\
\hline
x
&
Emission weighted temperature using \(L_X\)
\\
\hline
x\_cont
&
Emission weighted temperature using \(L_{X, cont}\)
\\
\hline
mass
&
Mass weighted temperature
\\
\hline
spec
&
Spectroscopic temperature (See \sphinxhref{https://iopscience.iop.org/article/10.1086/500121}{Vikhlinin, 2006})
\\
\hline
spec\_corr
&
Spectroscopic (core-corrected) temperature
\\
\hline
x\_corr
&
Emission weighted (core-corrected) temperature using \(L_X\)
\\
\hline
x\_corr\_cont
&
Emission weighted (core-corrected) temperature using \(L_{X, cont}\)
\\
\hline
mass\_corr
&
Mass weighted (core-corrected) temperature
\\
\hline
\end{tabulary}
\par
\sphinxattableend\end{savenotes}

S (calculated within a thin spherical shell with thickness = 1 kpc):


\begin{savenotes}\sphinxattablestart
\centering
\begin{tabulary}{\linewidth}[t]{|T|T|}
\hline
\sphinxstyletheadfamily 
Key name
&\sphinxstyletheadfamily 
Description
\\
\hline
500
&
Entropy of the shell at \(R_{500}\)
\\
\hline
2500
&
Entropy of the shell at \(R_{2500}\)
\\
\hline
\end{tabulary}
\par
\sphinxattableend\end{savenotes}


\subsection{Examples}
\label{\detokenize{Description:examples}}
Comming soon… For now, see “Shao’s Final Tutorial.ipynb” for details.


\chapter{Indices and tables}
\label{\detokenize{index:indices-and-tables}}\begin{itemize}
\item {} 
\DUrole{xref,std,std-ref}{genindex}

\item {} 
\DUrole{xref,std,std-ref}{modindex}

\item {} 
\DUrole{xref,std,std-ref}{search}

\end{itemize}


\renewcommand{\indexname}{Python Module Index}
\begin{sphinxtheindex}
\let\bigletter\sphinxstyleindexlettergroup
\bigletter{m}
\item\relax\sphinxstyleindexentry{modules.calculate\_R}\sphinxstyleindexpageref{calculate_R:\detokenize{module-modules.calculate_R}}
\item\relax\sphinxstyleindexentry{modules.cosmology}\sphinxstyleindexpageref{cosmology:\detokenize{module-modules.cosmology}}
\item\relax\sphinxstyleindexentry{modules.gas\_properties}\sphinxstyleindexpageref{gas_properties:\detokenize{module-modules.gas_properties}}
\item\relax\sphinxstyleindexentry{modules.halo\_analysis}\sphinxstyleindexpageref{halo_analysis:\detokenize{module-modules.halo_analysis}}
\item\relax\sphinxstyleindexentry{modules.prepare\_pyatomdb}\sphinxstyleindexpageref{prepare_pyatomdb:\detokenize{module-modules.prepare_pyatomdb}}
\item\relax\sphinxstyleindexentry{modules.X\_properties}\sphinxstyleindexpageref{X_properties:\detokenize{module-modules.X_properties}}
\end{sphinxtheindex}

\renewcommand{\indexname}{Index}
\printindex
\end{document}